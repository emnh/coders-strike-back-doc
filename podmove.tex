\documentclass[main.tex]{subfiles}
\begin{document}

\section{Pod movement}

\subsection{Expert rules for pod movement}

On each turn the pods movements are computed this way:
\begin{enumerate}
\item Rotation: the pod rotates to face the target point, with a maximum of 18 degrees (except for the 1rst round).
\item Acceleration: the pod's facing vector is multiplied by the given thrust value. The result is added to the current speed vector.
\item Movement: The speed vector is added to the position of the pod. If a collision would occur at this point, the pods rebound off each other.
\item Friction: the current speed vector of each pod is multiplied by 0.85
\item The speed's values are truncated and the position's values are rounded to the nearest integer.
\end{enumerate}

\par
Collisions are elastic. The minimum impulse of a collision is 120.
A boost is in fact an acceleration of 650. The number of boost available is common between pods.
If no boost is available, the maximum thrust is used.
A shield multiplies the Pod mass by 10.
The provided angle is absolute. 0 degrees means facing EAST while 90 degrees means facing SOUTH.

\newcommand{\cpx}[1]{\checkpoints[0][#1][0]}
\newcommand{\cpy}[1]{\checkpoints[0][#1][1]}

\subsection{Example pod movement}

\par
% TODO: output map once again
Let's see an example of pod movement. Let's work with map 1:

\begin{center}
\begin{tikzpicture}
\drawmap{1}
\end{tikzpicture}
\captionsetup{hypcap=false}
\captionof{figure}{Map Number: 1}\label{exmap}%
\end{center}

Assume the pod is starting at checkpoint 0 at
$(\print{\cpx{0}}, \print{\cpy{0}})$ and wants to move towards checkpoint 1 at
$(\print{\cpx{1}}, \print{\cpy{1}})$.
We are able to choose the starting rotation so let's assume starting angle towards checkpoint 1.

\subsubsection{Calculating the angle}

\par
To calculate the angle of a vector we can use the \href{https://github.com/inoryy/csb-ai-starter/blob/master/main.cpp#L325}{getAngle} method from inoryy's starter bot.

\lstset{caption={getAngle method}}
\begin{lstlisting}
inline float get_angle(Point* p) {
    float d = this->dist(p);
    float dx = (p->x - x) / d;
    float dy = (p->y - y) / d;

    float a = acos(dx) * 180 / M_PI;

    if (dy < 0) {
        a = 360 - a;
    }

    return a;
}
\end{lstlisting}

\pgfmathsetmacro{\xone}{\cpx{0}}
\pgfmathsetmacro{\xtwo}{\cpx{1}}
\pgfmathsetmacro{\yone}{\cpy{0}}
\pgfmathsetmacro{\ytwo}{\cpy{1}}
\newcommand{\dist}[0]{\fpeval{sqrt((\xtwo - \xone)^2 + (\yone - \ytwo)^2)}}
\newcommand{\dx}[0]{\fpeval{(\xtwo - \xone) / \dist}}
\newcommand{\dy}[0]{\fpeval{(\ytwo - \yone) / \dist}}
\newcommand{\acosangle}[0]{\fpeval{acos(\dx) * 180.0 / pi}}
\newcommand{\podangle}[0]{\fpeval{\dy < 0 ? (360.0 - \acosangle) : \acosangle}}
\newcommand{\tx}[0]{\fpeval{cos(\podangle * pi / 180)}}
\newcommand{\ty}[0]{\fpeval{sin(\podangle * pi / 180)}}
\newcommand{\vxzero}{\fpeval{\tx * \maxthrust}}
\newcommand{\vyzero}{\fpeval{\ty * \maxthrust}}

\newcommand{\num}[1]{\fpeval{trunc(#1,0)}}
\newcommand{\numf}[1]{\fpeval{trunc(#1,4)}}

% \newcommand{\dist}[0]{\fpeval{\xone}}

$$ x_1 = \xone $$
$$ y_1 = \yone $$
$$ x_2 = \xtwo $$
$$ y_2 = \ytwo $$
$$ d = \sqrt{(x_2 - x_1)^2 + (y_2 - y_1)^2} $$
$$ d = \sqrt{(\xtwo - \xone)^2 + (\yone - \ytwo)^2} $$
$$ d = \sqrt{(\xtwo - \xone)^2 + (\yone - \ytwo)^2} $$
$$ d \approx \num{\dist} $$
$$ d_x = \frac{(\xtwo - \xone)}{\num\dist} = \numf\dx $$
$$ d_y = \frac{(\ytwo - \yone)}{\num\dist} = \numf\dy $$
$$ \angle_{acos} = \frac{\arccos{d_x} * 180}{\pi} = \num\acosangle \degree $$
% $$ \angle_{pod} = d_y < 0 ? 360 - acosAngle : acosAngle = \num\podAngle $$

$$
  X=\left\{
  \begin{array}{@{}ll@{}}
    360 - \angle_{acos}, & \text{if}\ d_y < 0 \\
    \angle_{pod}, & \text{otherwise}
  \end{array}\right.
  = \angle_{pod} = \num\podangle \degree
$$

\subsubsection{Calculating the velocity}
\par
Now we will give the pod a full thrust and no turning for the first turn and see where it ends up after a few turns.
Give a thrust $t$ the new velocity vector is given by the recurrence:

$$ v_{x_n} = v_{x_{n - 1}} + \cos(\angle_{pod_{n}}) * t $$
$$ v_{y_n} = v_{y_{n - 1}} + \sin(\angle_{pod_{n}}) * t $$

The max thrust is \maxthrust. We will give the pod full throttle! :)
The pod is at rest at the start of the race.
So with this formula we can compute $v_{x_0}$ and $v_{y_0}$.

$$ t_{max} = \maxthrust $$
$$ t_x = \cos(\frac{\angle_{pod}*\pi}{180}) = \numf\tx $$
$$ t_y = \sin(\frac{\angle_{pod}*\pi}{180}) = \numf\ty $$
$$ v_{x_0} = t_x * t_max = \numf\tx * \maxthrust = \numf\vxzero $$
$$ v_{y_0} = t_y * t_max = \numf\ty * \maxthrust = \numf\vyzero $$

Actually now we should take into account friction of $\friction $ and the fact that speed is truncated at end of turn, so the new velocity vector becomes:
$$ f = \friction $$
$$ v_{x_1} = \trunc(v_{x_0} * f) + \cos(\angle_{pod}) * t $$
$$ v_{y_1} = \trunc(v_{y_0} * f)+ \sin(\angle_{pod}) * t $$

\subsubsection{Moving the pod}
\par
Using the formula including friction and assuming the pod doesn't accelerate after first turn (don't worry we'll add it later) we can compute the trajectory of the pod:

\newcommand{\stepPod}[3]{
\newcount{\xn}
\newcount{\yn}
\newcount{\vx}
\newcount{\vy}
\newcount{\oxn}
\newcount{\oyn}
\newcount{\anglecounter}
\pgfmathsetcount{\vx}{\vxzero}
\pgfmathsetcount{\vy}{\vyzero}
\pgfmathsetcount{\xn}{\xone}
\pgfmathsetcount{\yn}{\yone}
\pgfmathsetcount{\oxn}{\xn}
\pgfmathsetcount{\oyn}{\yn}
\pgfmathsetcount{\anglecounter}{\podangle}

% \newcommand{\tx}[0]{\fpeval{cos(\podangle * pi / 180)}}
% \newcommand{\ty}[0]{\fpeval{sin(\podangle * pi / 180)}}

\foreach \i [evaluate={\i as \j using {\i - 1}}] in {1,...,#1} {
    \pgfmathtruncatemacro{\k}{\j}
    #3
    \pgfmathsetcount{\global\oxn}{\xn}
    \pgfmathsetcount{\global\oyn}{\yn}
    \pgfmathsetcount{\global\vx}{\fpeval{trunc(\vx * 0.85, 0)} + \tx * #2}
    \pgfmathsetcount{\global\vy}{\fpeval{trunc(\vy * 0.85, 0)} + \ty * #2}
    \pgfmathsetcount{\global\xn}{\fpeval{round(\xn + \vx, 0)}}
    \pgfmathsetcount{\global\yn}{\fpeval{round(\yn + \vy, 0)}}
}
}
\stepPod{6}{0}{
$$ (v_{x_\k}, v_{y_\k}) = (\the\vx, \the\vy) $$
$$ (p_{x_\k}, p_{y_\k}) = (\the\xn, \the\yn) $$
}

\par
And then we see the pod moving like this (not so far at all):

\begin{center}
\begin{tikzpicture}
\drawmap{1}
\stepPod{6}{0}{
\draw[->] (\sx{\oxn}, \sy{\oyn}) -- (\sx{\xn}, \sy{\yn});
}
\end{tikzpicture}
\captionsetup{hypcap=false}
\captionof{figure}{Pod Trajectory}\label{exmap}%
\end{center}

\par
So let us uphold a constant max thrust for the duration of the 6 turns this time and see what happens. We get:

\begin{center}
\begin{tikzpicture}
\drawmap{1}
\stepPod{6}{\maxthrust}{
\draw[->] (\sx{\oxn}, \sy{\oyn}) -- (\sx{\xn}, \sy{\yn});
\draw[line width=0.5pt] (\sx{\oxn}, \sy{\oyn}) -- (\sx{\xn}, \sy{\yn});
% [postaction={decorate,decoration={markings,mark=at position 0.5 with {\arrow[black,line width=1.5pt]{>}}}}]
}
\end{tikzpicture}
\captionsetup{hypcap=false}
\captionof{figure}{Pod Trajectory}\label{exmap}%
\end{center}

\par
We are more than halfway there this time :) . So we've seen a bit how the pod moves.
But now how can we devise a strategy for adjusting the angle to move from checkpoint to checkpoint?
A popular strategy that will get you to Gold league is -3vel, that is, subtract 3 times the current
velocity to get the target for steering which you can output directly by CG bot, by the following formula.
We already used $t$ for thrust so let's call the target $u$.

$$ u_x = p_x - 3 * v_x $$
$$ u_y = p_y - 3 * v_y $$


\end{document}