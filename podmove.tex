\documentclass[main.tex]{subfiles}
\begin{document}

\section{Pod movement}

\subsection{Expert rules for pod movement}

On each turn the pods movements are computed this way:
\begin{enumerate}
\item Rotation: the pod rotates to face the target point, with a maximum of 18 degrees (except for the 1rst round).
\item Acceleration: the pod's facing vector is multiplied by the given thrust value. The result is added to the current speed vector.
\item Movement: The speed vector is added to the position of the pod. If a collision would occur at this point, the pods rebound off each other.
\item Friction: the current speed vector of each pod is multiplied by 0.85
\item The speed's values are truncated and the position's values are rounded to the nearest integer.
\end{enumerate}

\par
Collisions are elastic. The minimum impulse of a collision is 120.
A boost is in fact an acceleration of 650. The number of boost available is common between pods.
If no boost is available, the maximum thrust is used.
A shield multiplies the Pod mass by 10.
The provided angle is absolute. 0 degrees means facing EAST while 90 degrees means facing SOUTH.

\newcommand{\cpx}[1]{\checkpoints[0][#1][0]}
\newcommand{\cpy}[1]{\checkpoints[0][#1][1]}

\subsection{Example pod movement}

\par
Let's see an example of pod movement. Let's work with map 1 in figure \ref{map1}. Assume the pod is starting at checkpoint 0 at
$(\print{\cpx{0}}, \print{\cpy{0}})$ and wants to move towards checkpoint 1 at
$(\print{\cpx{1}}, \print{\cpy{1}})$.
We are able to choose the starting rotation so let's assume starting angle towards checkpoint 1.

\subsubsection{Calculating the angle}

\par
To calculate the angle of a vector we can use the \href{https://github.com/inoryy/csb-ai-starter/blob/master/main.cpp#L325}{getAngle} method from inoryy's starter bot.

\lstset{caption={getAngle method}}
\begin{lstlisting}
inline float get_angle(Point* p) {
    float d = this->dist(p);
    float dx = (p->x - x) / d;
    float dy = (p->y - y) / d;

    float a = acos(dx) * 180 / M_PI;

    if (dy < 0) {
        a = 360 - a;
    }

    return a;
}
\end{lstlisting}

% \newcommand{\xone}{\pgfmathparse{\cpx[0]}}
% \newcommand{\xtwo}{\pgfmathparse{\cpy[0]}}
% \newcommand{\yone}{\pgfmathparse{\cpx[1]}}
% \newcommand{\ytwo}{\pgfmathparse{\cpy[1]}}
% % \newcommand{\dist}[0]{sqrt(({\xtwo{}} - {\xone{}})^2 + ({\yone{}} + {\ytwo{}})^2)}
% \newcommand{\dist}[0]{\fpeval{\xone}}

% $$ x_1 = \print{\cpx{0}} $$
% $$ y_1 = \print{\cpy{0}} $$
% $$ x_2 = \print{\cpx{1}} $$
% $$ y_2 = \print{\cpy{1}} $$
% $$ d = \sqrt{(x_2 - x_1)^2 + (y_2 - y_1)^2} = {\dist} $$

\end{document}